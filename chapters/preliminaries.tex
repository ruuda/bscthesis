% !TEX root = ../thesis.tex

\chapter{Preliminaries}
\label{chap:preliminaries}
There are several topological spaces that play a key role
when describing properties of the Hopf map $h : S^3 \to S^2\!$.
These include of course the domain $S^3$ and codomain $S^2\!$,
which are traditionally defined as subspaces of $\R^4$ and $\R^3$ respectively.
As we will see later,
it is useful to consider $S^3$ and $S^2$ as quotient spaces of $\C^2 \setminus \set{0}$
or subspaces of the quaternion algebra $\H$ instead.
Both $\C^2$ and $\H$ are isomorphic to $\R^4$ as a real vector space,
but their additional structure sheds light on various properties of the Hopf map.
We will therefore briefly examine these spaces before defining the Hopf map.
In later chapters we will explore the differentiable structure of these spaces,
but for now we focus on their topological properties.
Furthermore, group actions are used extensively throughout this chapter,
so we quickly recall some of the terms involved.

\section{Definitions}

\definition
Let $X$ be a set with additional structure,
such as a vector space or a topological space.
The \emph{automorphism group} of $X$,
denoted $\Aut(X)$ is the group of bijections $X \to X$
that preserve its structure.
More formally we can say that $X$ is an object in a concrete category $\mathcal{A}$,
a category equipped with a faithful functor $\mathcal{A} \to \mathbf{Sets}$,
the forgetful functor.
The automorphism group is the group of invertible morphisms $X \to X$
in this category.

\example
For a group $G$, $\Aut(G)$ is the group of group isomorphisms $G \to G$.
For a module $M$ over a ring $R$,
or for a vector space $V$ over a field $F$,
the automorphism group consists of $R$-linear bijections $M \to M$
and $F$-linear bijections $V \to V$ respectively.
For a topological space $X$,
$\Aut(X)$ is the group of homeomorphisms $X \to X$.
For sets, the automorphism group is simply the group of bijections
from the set to itself.

\definition
Let $G$ be a group and $X$ a set with additional structure.
A \emph{group action} of $G$ on $X$ is a group homomorphism $\phi : G \to \Aut(X)$.
If $X$ is an object in the category $\mathcal{A}$ \,(e.g. the category of vector spaces,
topological spaces, etc.),
we say $G$ acts on $X$ in this category.
Given an element $x \in X$ and $g \in G$,
we will often write $g \cdot x$ for the element $\phi(g)(x)$.

In the definition above,
the group $G$ acts \emph{from the left} on $X$.
For $g, h \in G$ and $x \in X$,
we have $(gh) \cdot x = g \cdot (h \cdot x)$.
Sometimes we encounter a natural \emph{antihomomorphism} $G \to \Aut(X)$.
In this case,
we say that $G$ acts \emph{from the right} on $X$ in the category $\mathcal{A}$,
and to make the notation more natural we write $x \cdot g$ instead of $g \cdot x$,
so that $x \cdot (gh) = (x \cdot g) \cdot h$.

\definition
Let $G$ and $X$ be as before, and $x \in X$.
The \emph{orbit} of $x$ is the set
\[ Gx = \set{ g \cdot x \mid g \in G } \]
Having the same orbit defines an equivalence relation on $X$,
and the quotient with respect to this relation is called the \emph{orbit space},
written $X/G$.

\definition
Let $G$ and $X$ be as before, and $x \in X$.
The \emph{stabiliser} of $x$ is the subgroup
\[ G_x = \set{ g \in G \mid g \cdot x = x } \]

\proposition[prop:open-quotient-map]
Let $X$ be a topological space, $G$ a group that acts on $X$.
When $X/G$ is endowed with the quotient topology,
the quotient map $q : X \surj X/G$ is an open map.

\proof
Let $U \subseteq X$ be open,
define $V = q(U)$.
$V$ is open if and only if $q^{-1}(V)$ is open
by definition of the quotient topology.
We have
\[ q^{-1}(V)
 = \bigcup_{g \in G} g \cdot U \]
Because the automorphisms of the group action are homeomorphisms,
they are open maps,
so $g \cdot U$ is open for all $g \in G$.
Hence, $q^{-1}(V)$ is the union of open sets,
so it is open.
\qed

\definition
A \emph{topological group} is a group $G$ that is also a Hausdorff space,
such that the map $G \times G \to G$, $(g, h) \mapsto gh^{-1}$ is continuous
when $G \times G$ is endowed with the product topology.
This is equivalent to the statement that multiplication and inversion are continuous;
see for example \parencite[ch.~\textsc{iii}, \S~1.1]{bourbaki1971} or \parencite[p.~276]{szekeres2004}.

\definition
Let $G$ be a topological group and $X$ a topological space,
such that $G$ acts on $X$ in the category of sets.
The action is said to be \emph{continuous} if
\[ G \times X \longto X,
\quad (g, x) \longmapsto g \cdot x \]
is a continuous map.
It follows immediately from this definition that
$x \mapsto g \cdot x$ is a homeomorphism for all $g \in G$ when $G$ acts continuously on $X$,
therefore $G$ acts on $X$ in the category of topological spaces;
the group homomorphism $G \to \Aut(X)$ is actually a group homomorphism $G \to \Homeo(X)$.

\theorem{Universal property of the quotient topology}[thm:universal-property-quotient-topology]
Let $X$ and $Y$ be topological spaces and $\sim$ an equivalence relation on $X$.
Denote by $q : X \surj X\modsim$ the quotient map.
Let $f : X \to Y$ be a continuous map such that
for all $x, y \in X$ it holds that $x \sim y$ implies $f(x) = f(y)$.
(Such $f$ is said to be \emph{compatible} with the equivalence relation.)
Then there exists a unique continuous map $g : X\modsim \to Y$ that makes the following diagram commute:
\vspace{-\parskip}
\begin{center}
\begin{tikzcd}[column sep = small] &
X        \ar[dl, two heads, swap, "q"]
         \ar[dr, "f"] & \\
X\modsim \ar[rr, dashed, swap, "\exists_! g"] & & Y
\end{tikzcd}
\end{center}

\proof
See for example \parencite[ch.~\textsc{i}, \S~3.4]{bourbaki1971}.

\proposition
Let $G_1$ and $G_2$ be groups, and $X$ a set.
Suppose that $G_1 \times G_2$ acts on $X$ in the category of sets.
This implies that the subgroups $G_1$ and $G_2$ act on $X$ individually as well.
Then the following holds:
\begin{enumerate}
\item There is a natural action of $G_1$ on $X / G_2$ in the category of sets.
\item If $X$ is a topological space and $G_1 \times G_2$ acts in the category of topological spaces,
      $G_1$ acts on $X / G_2$ in this category.
\item If $G_1$ and $G_2$ are topological groups such that $G_1 \times G_2$ acts continuously on $X$,
      then $G_1$ acts continuously on $X / G_2$.
\end{enumerate}

\proof
Let $x \in X$ such that $[x] \in X / G_2$, and $g \in G_1$.
Define $g \cdot [x] = [g \cdot x]$.
We have to show that this action is well-defined.
Suppose that $[x] = [y]$ for some $y \in X$.
Then there exists an $h \in G_2$ such that $x = h \cdot y$.
Because $h$ and $g$ commute in $G_1 \times G_2$,
we have
\[ g \cdot x = g \cdot (h \cdot y) = (g, h) \cdot y = h \cdot (g \cdot y) \]
Thus, we find $[g \cdot x] = [g \cdot y]$.
That this defines a homomorphism $G_1 \to \Aut(X / G_2)$ follows
from the fact that $G_1 \to \Aut(X)$ is a homomorphism.
This proves statement \textbbf{i}.

Suppose that $X$ is a topological space
and $G_1 \times G_2$ acts on $X$ in the category of topological spaces.
Let $g \in G_1$, then $g$ induces a homeomorphism $\phi : X \to X$,
and a bijection $\psi : X / G_2 \to X / G_2$.
Denote by $q : X \surj X / G_2$ the quotient map,
then $q \circ \phi$ is a continuous map $X \to X / G_2$
that satisfies $\psi \circ q = q \circ \phi$ due to statement \textbbf{i}.
This means that $q \circ \phi$ is compatible with the quotient map,
so by the universal property of the quotient topology (theorem \ref{thm:universal-property-quotient-topology}),
there exists a unique continuous map $\psi'$ such that $q \circ \phi = \psi' \circ q$.
Uniqueness implies that $\psi' = \psi$,
therefore $\psi$ is continuous.
The same argument applies to $\psi^{-1}$,
hence $\psi$ is a homeomorphism.
This shows that $G_1$ acts on $X / G_2$ in the category of topological spaces,
which proves statement \textbbf{ii}.

To prove statement \textbbf{iii},
we will use the following diagram:
\begin{center}
\begin{tikzcd}[row sep = large]
&
X \ar[dr, two heads, "q"] & \\
G_1 \times X           \ar[ur, "a"]
                       \ar[r, "f"]
                       \ar[dr, two heads, swap, "r"] &
G_1 \times {(X / G_2)} \ar[r, "T"] &
X / G_2 \\ &
{(G_1 \times X)} / G_2 \ar[u, dashed, "\exists_! \phi"]
                       \ar[ur, dashed, swap, "\exists_! \psi"] &
\end{tikzcd}
\end{center}
The map $a : G_1 \times X \to X$,
$(g, x) \mapsto g \cdot x$
is continuous because it is the restriction of $G_1 \times G_2 \times X \to X$
that is continuous by assumption.
Let $q: X \surj X / G_2$ denote the quotient map.
Define $f : G_1 \times X \to G_1 \times {(X / G_2)}$ as $f = (\id, q)$.
This map is continuous because both of its coordinates are.
(See proposition 1 of \parencite[ch.~\textsc{i}, \S~4.1]{bourbaki1971}.)
Furthermore $f$ is open,
for $\id$ and $q$ are open. (See proposition~\ref{prop:open-quotient-map}.)
Let $G_2$ act on $G_1 \times X$ by $h \cdot (g, x) = (g, h \cdot x)$
where $g \in G_1, h \in G_2, x \in X$,
and let $r$ denote the quotient map.
$f$ is compatible with $r$,
so by the universal property of the quotient topology (theorem \ref{thm:universal-property-quotient-topology}),
there exists a unique continuous map
$\phi$ that makes the bottom left triangle of the diagram commute.
The map is given by $[g, x] \mapsto (g, [x])$
and its inverse is given by $(g, [x]) \mapsto [g, x]$.
An open set in $(G_1 \times X) / G_2$
is the image under $r$ of an open set in $G_1 \times X$,
so from commutativity it follows that $\phi$ is an open map.
Hence, $\phi$ is a homeomorphism.

On the top of the diagram, we have the map $q \circ a : G_1 \times X \to X / G_2$,
given by $(g, x) \mapsto [g \cdot x]$.
As the composition of continuous maps it is continuous,
and it is compatible with $r$.
Thus, by the universal property of the quotient topology,
there exists a unique continuous map $\psi$
such that $\psi \circ r = q \circ a$.
Composing with $\phi^{-1}$, we find that the map
\[ T : G_1 \times {(X / G_2)} \longto X / G_2,
\quad (g, [x]) \longmapsto [g \cdot x] \]
is continuous,
which proves claim \textbbf{iii}.
Furthermore, the above diagram commutes.
\qed

\theorem[thm:quotient-map-factors]
Let $G_1$ and $G_2$ be groups and $X$ a topological space
such that $G_1 \times G_2$ acts on $X$.
Then $X / (G_1 \times G_2)$ is canonically homeomorphic to $(X / G_1) / G_2$.
In particular, the quotient map $X \surj X / (G_1 \times G_2)$ factors over $X / G_1$.

\proof
Let
$q_1 : X \surj X / G_1$,
$q_2 : (X / G_1) \surj (X / G_1) / G_2$,
and $q_{12} : X \surj X / (G_1 \times G_2)$ denote the quotient maps.
Then we have the following commutative diagram:
\begin{center}
\begin{tikzcd}[row sep = huge, column sep = large]
 X                    \ar[r,  two heads, "q_1"]
                      \ar[d,  two heads, "q_{12}"] &
 X / G_1              \ar[d,  two heads, "q_2"]
                      \ar[dl, dashed, bend right = 7, swap, "\phi_1"] \\
 X / (G_1 \times G_2) \ar[r,  dashed, bend left = 13, "\phi_{12}"] &
(X / G_1) / G_2       \ar[l,  dashed, bend left = 13, "\phi_2"]
\end{tikzcd}
\end{center}
The map $q_2 \circ q_1$ is continuous and compatible with $q_{12}$,
so by the universal property of the quotient topology (theorem \ref{thm:universal-property-quotient-topology})
there exists a unique continuous map $\phi_{12}$ that makes the diagram commute.
Because $q_{12}$ is compatible with $q_1$,
there exists a unique continuous map $\phi_1$ such that $q_{12} = \phi_1 \circ q_1$.
It follows that $\phi_1$ is compatible with $q_2$,
so there exists a unique continuous map $\phi_2$ that makes the diagram commute.
Now we see that $\phi_{12}$ and $\phi_2$ are continuous inverses of one another,
hence $X / (G_1 \times G_2)$ and $(X / G_1) / G_2$ are homeomorphic.
\qed

\subsection*{Physical interpretation}
Groups are prevalent in mathematics.
In physics, groups are often encountered in the context of symmetries.
In that case one may think of a group as a set of transformations of a system,
transformations under which a certain property is invariant.
For instance, angular momentum is invariant under rotation of space,
and four-momentum is invariant under Lorentz transformations.
A \emph{group action} generalises this idea.
Elements of the group induce a transformation of a system.
By applying all possible transformations to a point,
we obtain the \emph{orbit} of a point.
For instance, when we let the Lorentz group act on Minkowski space,
the orbit of a timelike vector is all of the light cone (past and future).
Often, a group encodes transformations that we are \emph{not} interested in.
The \emph{orbit space} is what remains if we consider points that differ
by such a transformation to be equal.
For example, the orbit space of the Lorentz group action on Minkowski space
consists of four elements:
the origin, the class of null (or light-like) vectors,
the class of timelike vectors,
and the class of spacelike vectors.
The \emph{stabiliser} of a point is the subgroup of transformations under which the point is invariant.

Topology is the branch of mathematics that studies abstract properties of space.
It gives us the tools to study properties that do not depend
on exact distances, but rather on overall shape.
For instance, one would like to think of a garden hose
as a one dimensional system where water can move back and forth,
regardless of how the hose is bent or twisted.
Topology allows us to ignore the bending and twisting.
Virtually all spaces that occur in physics are topological spaces:
$\R^3\!$, Minkowski space, Hilbert spaces, etc.
Often these spaces have additional structure
such as a metric or inner product,
but many properties can be derived from the topology alone.
An important example of such a property
is \emph{continuity} of a map between topological spaces,
a notion that is prevalent throughout physics.
Many of the groups encountered in this thesis happen to have a natural topology as well.
In this case, an action on another topological space can be \emph{continuous}.
The definition given in this section codifies our intuition:
if two group elements that are near act on a point,
the resulting points should be near as well.

\section{Projective space}
\label{sec:projective-space}
It is possible to identify $\R^4$ and $\C^2$ as four-dimensional real vector spaces,
by identifying the standard basis $(e_1, e_2, e_3, e_4)$ with the basis $((1, 0), (i, 0), (0, 1), (0, i))$.
The space $\C^2 \setminus \set{0}$ will be prevalent in the rest of this section,
so we introduce a shorthand notation. Furthermore, we embed $S^1$ in $\C$.

\definition
$\CZ = \C^2 \setminus \set{0}$.

\definition
The \emph{unit circle} is defined by
\[ S^1 = \set{ \, z \in \C \mid 1 = |\,z\,| \, } \]
This is a group under multiplication.

\definition[def:s3-real]
The \emph{three-sphere} is defined by
\[ S^3 = \set{ \, x \in \R^4 \mid 1 = \nsq{x} } \]
Here $\|{}\cdot{}\|$ denotes the regular Euclidean norm.
By identifying $\R^4$ with $\C^2$ as above,
we can consider $S^3$ to be a subset of $\CZ$.

Consider the multiplicative group $\Rpos$ of positive real numbers.
It acts continuously on $\C^2$ (in the category of real vector spaces) by scalar multiplication,
and this action can be restricted to $\CZ$ (in the category of sets).
This allows us to give an alternative definition of $S^3$ as a quotient:

\definition[def:s3-complex]
$\SC$ is the orbit space of $\CZ$ with respect to the $\Rpos$ action.
Denote by $r : \CZ \surj \SC$ the quotient map.
$\C^2$ is endowed with its regular topology induced by the Euclidean metric,
and $\SC$ is endowed with the quotient topology.

Intuitively, this definition is not that different from definition~\ref{def:s3-real}.
Every point $p$ at the three-sphere defines a ray from the origin through $p$.
This ray, except for the origin, is the orbit of $p$ under the $\Rpos$ action.
In other words, every orbit can be represented by a point at unit distance from the origin.
The quotient map $r$ corresponds to projection onto the sphere.

\proposition[prop:s3-equivalence]
$S^3$ and $\SC$ as defined in definition~\ref{def:s3-real} and \ref{def:s3-complex} are homeomorphic.

\proof
\newcommand*{\RZ}{{\R^4_{\,\circ}}}
Write $\R^4 \setminus \set{0} = \RZ$.
Let $i : S^3 \to \RZ$ be the inclusion,
and let $\phi : \R^4 \to \C^2$ be the vector space isomorphism
induced by the identification of the bases given earlier in this section.
The inclusion $i$ is continuous, and the restriction $\phi|_\RZ = \phi_\circ$ is a homeomorphism.
Therefore, the composition $\psi = r \circ \phi_\circ \circ i : S^3 \to \SC$ is continuous.
Consider the map
\vspace{-0.3\parskip}
\[ \CZ \longto S^3,
\quad x \longmapsto \phi_\circ^{-1} \left( \frac{x}{\|x\|} \right) \vspace{0.3\parskip} \]
which is continuous and compatible with $r$.
By the universal property of the quotient topology (theorem~\ref{thm:universal-property-quotient-topology}),
this map induces a unique continuous map $\psi^{-1} : \SC \to S^3$ that is the inverse of $\psi$.
Thus, $\psi$ is a homeomorphism.
\qed

Consider the multiplicative group $\C^*$ (the complex plane minus the origin).
It acts continuously on $\C^2$ (in the category of complex vector spaces) by scalar multiplication,
and this action can be restricted to $\CZ$.
This allows us to define the projective space:

\definition[def:complex-projective-line]
The \emph{complex projective line} $\PC$ is the orbit space of $\CZ$ with respect to the $\C^*$ action.
Denote by $q : \CZ \surj \PC$ the quotient map.
$\PC$ is endowed with the quotient topology.

Elements of $\PC$ are indicated by \emph{homogeneous coordinates}:
if $(z_1, z_2)  \in \C^2$ is nonzero,
then we write $(z_1 : z_2)$ for $q(z_1, z_2)$.
We can embed $\C$ in $\PC$ via $z \mapsto (z : 1)$.
The only point that is not reached in this manner is $(1 : 0)$.

\theorem[thm:s2-homeom-p1c]
There exists a homeomorphism between $S^2$ and $\PC$.

\proof
We will postpone the proof until section \ref{sec:hopf-quaternionic},
and prove this with the aid of quaternions in theorem~\ref{thm:hopf-map-equivalence}.
For an alternative proof, see \parencite[ch.~\textsc{viii}, \S~4.3]{bourbaki1974}.

The general linear group $\GLC$
of invertible complex $2 \times 2$ matrices
acts on $\C^2$ by matrix multiplication.
This induces a group action of $\GLC$ on $\CZ$.
Furthermore, the groups $\Rpos$, $S^1$, and $\C^*$ are isomorphic to subgroups of $\GLC$:
given an element $z \in \C^*\!$,
we can identify it with the matrix
\[ \begin{pmatrix} z & 0 \\ 0 & z \end{pmatrix} \]
in the centre of $\GLC$.
$\C^*$ is isomorphic to the direct product $\Rpos \times S^1$:
this is the decomposition of a complex number into its modulus and argument.
It follows that $S^1$ and $\Rpos$ are central in $\GLC$, because their elements correspond to scalar matrices.
Consequently, $S^1$ and $\Rpos$ are normal in $\GLC$.

\subsection*{Informal summary}
The \emph{projective space} $\PC$ is a construction with several interpretations.
For starters, $\PC$ can be thought of as $\C$ with one extra point,
a point “at infinity”.
This allows us to talk about $z_1/z_2$ even when $z_2$ is zero.
Instead of $z_1/z_2$, we write $(z_1 : z_2)$, called \emph{homogeneous coordinates}.
Secondly, theorem \ref{thm:s2-homeom-p1c} tells us that $\PC$ can be thought of as the unit sphere $S^2\!$.
(In fact, $\PC$ is sometimes called the \emph{Riemann sphere}.)
A \emph{homeomorphism} between two spaces
is a function, both one-to-one and onto,
that preserves all topological properties.
From a topological point of view, $\PC$ and $S^2$ are the same space.
This means that when we formulate the Hopf map later on
— a function from $S^3$ to $S^2$ —
we can express it as a function to $\PC$.
This expression is significantly simpler than the one involving Cartesian coordinates on $S^2\!$.

\section{Quaternions}
\label{sec:quaternions}

\definition
The \emph{quaternion algebra} $\H$ is
the real noncommutative algebra with basis $(1, i, j, k)$.
Multiplication is given by the identities
\[ i^2 = j^2 = k^2 = -1,
\quad i\!j =  k, \ jk =  i, \ ki =  j,
\quad j  i = -k, \ kj = -i, \ ik = -j \]
and $1$ commutes with all elements.
In particular, $\H$ is a ring and a four-dimensional real vector space.
Analogously to complex numbers,
this algebra has an involution $\overline{\raisebox{0pt}[0.5em]{${}\cdot{}$}}$
called \emph{conjugation}
that flips the sign of the $i$, $j$, and $k$ components.

\definition
The \emph{trace} is the map $\Tr : \H \to \R,\ q \mapsto q + \overline{q}$.
Because the imaginary parts cancel, the trace of a quaternion is real.
Furthermore, the trace is $\R$-linear.

The reals commute with all quaternions,
so $\Tr(q)$ commutes with $q$ for all $q \in \H$.
Because $\overline{q} = \Tr(q) - q$,
it follows that $q$ and $\overline{q}$ commute.

\definition
The standard inner product on $\H$ is given by
\[ \inp{{}\cdot{}}{{}\cdot{}} : \H \times \H \longto \R,
\quad (p, q) \longmapsto \tfrac{1}{2} \Tr(p\overline{q}) = \tfrac{1}{2}(p\overline{q} + q \overline{p}) \]
Symmetry is clear from the definition,
and bilinearity follows from the linearity of the trace.
For positive definiteness,
remark that for $q = a + bi + cj + dk$,
we have $q \overline{q} = a^2 + b^2 + c^2 + d^2\!$.
Therefore $\inp{q}{q} \geq 0$,
and $\inp{q}{q} = 0 \implies a = b = c = d = 0 \implies q = 0$.

\definition
The \emph{norm} of $q \in \H$ is given by $\nsq{q} = q \overline{q}$.
Because $q$ and $\overline{q}$ commute,
$q\overline{q} = \smallfrac{1}{2}(q \overline{q} + \overline{q} q)$,
so the norm is induced by the inner product.
This norm coincides with the Euclidean norm on $\H$
as real vector space with orthonormal basis $(1, i, j, k)$.
Therefore,
$\H$ with the topology induced by the norm is homeomorphic to $\R^4\!$.

Because conjugation reverses the order of multiplication,
the norm is multiplicative:
for $p, q \in \H$, we have
\[ \nsq{pq}
 = (pq)\overline{(pq)}
 = p \, q \overline{q} \, \overline{p}
 = p \nsq{q} \, \overline{p}
 = \nsq{q} p \overline{p}
 = \nsq{q} \nsq{p} \]

Because $q \overline{q} = \nsq{q}\!$,
we have $q^{-1} = \overline{q} \, \|q\|^{-2}$ for $\|q\| \neq 0$.
Therefore, $\H$ is a division algebra:
every nonzero element has an inverse.

\proposition
$\H^* = \H \setminus \set{0}$ is a topological group.

\proof
Multiplication is continuous,
because for $p, q \in \H^*\,$,
the components of the product $pq$
can be written as a polynomial in the components of $p$ and $q$.
Inversion is continuous,
because the components of $q^{-1}$ are rational functions of the components of $q$,
which do not vanish because $\nsq{q} \neq 0$.
See also \parencite[ch.~\textsc{viii}, \S~1.4]{bourbaki1974}.
\qed

With this machinery,
we can give a quaternionic definition of $S^3$ and $S^2\!$.
Whereas the definitions in section \ref{sec:projective-space} emphasise
how $S^3$ and $S^2$ are quotients with respect to a group action,
the quaternionic definitions emphasise the group structure on the three-sphere itself,
and the action of $S^3$ on $S^2\!$.

Let us revisit the three-sphere as defined in definition~\ref{def:s3-real}.
By identifying $\H$ with $\R^4$ as a normed real vector space
via the basis given earlier in this section,
we can consider $S^3$ to be a subset of $\H$,
the set of quaternions with unit norm:
\[ S^3 = \set{ q \in \H \mid 1 = \nsq{q} } \]
This set is closed under multiplication due to the multiplicativity of the norm,
and it contains $1$.
Therefore, this is a subgroup of $\H^*\!$.
We can embed $S^2$ in $S^3\!$,
but in $\R^4$ there is no preferred way of doing so.
For quaternions, there is one natural choice:

\definition[def:s2-quaternion]
The \emph{two-sphere} $S^2 = \set{ q \in S^3 \mid \Tr(q) = 0 }$,
the set of pure imaginary quaternions with unit norm.
This definition coincides with the conventional definition of $S^2$
when $\R^3$ is identified with the subspace of $\H$ spanned by $i$, $j$, and $k$.
$S^2$ may alternatively be written as $\set{q \in S^3 \mid \inp{1}{q} = 0} = 1^\perp \cap S^3$.

The group $\H^*$ acts
on $\H$ in the category of $\R$-algebras via the following homomorphism:
\[ \phi : \H^* \longto \Aut(\H),
   \quad
   p \longmapsto (q \mapsto p q p^{-1}) \]
Because quaternion multiplication is continuous,
this is a continuous action.
By restriction to the subgroup $S^3\!$,
we get a continuous action of $S^3$ on $\H$.

\proposition[prop:invariant-inner-product]
The inner product on $\H$ is invariant under the action of $\H^*\!$.

\proof
Let $p \in \H^*\!$, $q_1, q_2 \in \H$,
then we have
\begin{align*}
   2 \, \inp{p \cdot q_1}{p \cdot q_2}
&= p q_1 p^{-1} \overline{p q_2 p^{-1}} + p q_2 p^{-1} \overline{p q_1 p^{-1}} \\
&= p q_1 \nsq{p^{-1}} \overline{q_2} \, \overline{p} + p q_2 \nsq{p^{-1}} \overline{q_1} \, \overline{p} \\
&= \nsq{p^{-1}} p (q_1  \overline{q_2} + q_2 \overline{q_1}) \overline{p} \\
&= \nsq{p^{-1}} \, \nsq{p} \, 2 \, \inp{q_1}{q_2} \\
&= 2 \, \inp{q_1}{q_2}
   \tag*{\qed}
\end{align*}

\corollary[cor:s3-action]
Identify $\R^3$ with the subspace of $\H$ spanned by $i$, $j$, and $k$.
Then $\R^3 = 1^\perp$ and $S^2 = 1^\perp \cap S^3$ are invariant under the action of $\H^*\!$,
which means $\H^*$ and its subgroup $S^3$ act continuously on $\R^3$ and $S^2\!$.

$\C$ is a commutative subring of~$\H$.
As real vector spaces with bases $(1, i)$ and $(1, i, j, k)$,
$\C$ can be identified with the subspace of $\H$ spanned by $1$ and $i$.
The stabiliser of $i \in \H$ consists of the nonzero elements that commute with $i$.
These elements are linear combinations of $1$ and $i$,
so we have $\H^*_i = \C^*$ and $S^3_i = S^1\!$.

\proposition[prop:s3-isom-su2c]
$S^3$ is isomorphic to $\SUC$, the group of unitary $2 \times 2$ matrices with determinant $1$.

\proof
Define the unitary matrices
\[            I = \begin{pmatrix}1 &  0 \\ 0 &  1\end{pmatrix}
\qquad \sigma_1 = \begin{pmatrix}0 &  1 \\ 1 &  0\end{pmatrix}
\qquad \sigma_2 = \begin{pmatrix}0 & -i \\ i &  0\end{pmatrix}
\qquad \sigma_3 = \begin{pmatrix}1 &  0 \\ 0 & -1\end{pmatrix} \]
These matrices are sometimes called the \emph{Pauli spin matrices}.
Let $\phi: \H \to \Mat(2 \times 2, \C)$ be the $\R$-linear extension of
\[    1 \longmapsto I,
\quad i \longmapsto i \sigma_1,
\quad j \longmapsto i \sigma_2,
\quad k \longmapsto i \sigma_3 \]
Let $\psi$ be the restriction of $\phi$ to $S^3\!$.
All matrices in the image of $\psi$ are unitary,
and a little computation shows that for $q \in S^3\!$,
$\det \psi(q) = 1$.
The matrices $I, i\sigma_1, i\sigma_2, i\sigma_3$
satisfy the same multiplication rules as $1, i, j, k$.
That is, $i\sigma_1 i\sigma_2 = i\sigma_3$, etc.
Therefore, $\psi$ is a group homomorphism $S^3 \to \SUC$.
This homomorphism is surjective (see \parencite[p.~173]{szekeres2004}),
and $1$ is the only element in its kernel.
Therefore, $\psi$ is an isomorphism.
\qed

\theorem[thm:s3-surj-so3r]
The map
\[ \phi: S^3 \longto \SOR,
\quad q \longmapsto (x \mapsto q \cdot x) \]
is a surjective group homomorphism with kernel $\set{\pm 1}$.
Here $x \in \R^3 \cong \Span(i, j, k)$.

\proof
The map $x \mapsto q \cdot x$ is linear,
and orthogonality follows from
the fact that the inner product is invariant under the action,
as shown in proposition~\ref{prop:invariant-inner-product}.
To show that $x \mapsto q \cdot x$ is not a reflection,
note that $\det : \textup{O}_3(\R) \to \R$ is a continuous map
(see for example \parencite[p.~281]{hatcher2002}).
We can express $\phi$ as a polynomial on all coordinates
when elements of $\SOR$ are written as matrices,
so $\phi$ is continuous.
By composition we get a continuous map $S^3 \to \set{\pm 1}$.
Because $S^3$ is connected, this map must be constant.
% (See for instance proposition~3.4.11 of \parencite[p.~92]{runde2005}.)
The determinant of $\id$ is $1$,
so all $q \in S^3$ induce an orthogonal map with positive determinant.

To show that the kernel of $\phi$ is $\set{\pm 1}$,
suppose that $q \in S^3$ is such that $q \cdot x = q x q^{-1} = x$ for all $x \in \R^3\!$.
Then $q$ commutes with all $x \in \R^3\!$,
so $q$ must be real.
Because $\nsq{q} = 1$, it follows that $q = 1$ or $q = -1$.

To prove surjectivity,
suppose that $\rho \in \SOR$ is an anticlockwise rotation
of $\alpha$ radians about an axis spanned by $u \in \R^3\!$,
where $\nsq{u} = 1$.
Then the quaternion $q = \cos(\smallfrac{1}{2}\alpha) + u \sin(\smallfrac{1}{2}\alpha)$
will map to $\rho$.
To see this,
note that all points on the axis of rotation are fixed points,
for $u$ commutes with $q$.
Furthermore, suppose that $v \in \R^3$ is such that $\inp{u}{v} = 0$.
Set $q_0 = \cos(\smallfrac{1}{2}\alpha)$ and $\vec{q} = u \sin(\smallfrac{1}{2}\alpha)$.
By using identities from \parencite[p.~157]{szekeres2004},
we find
\newcommand*{\vv}{v} % Could switch to v with an arrow, but to me it is just noise.
\newcommand*{\vq}{\vec{q}}
\begin{align*}
q \cdot v &= (q_0 + \vq) \vv (q_0 - \vq)
%      \\ &= (q_0 + \vq) (\inp{v}{\vq} + q_0 \vec{v} - \vec{v} \times \vq)
       \\ &= (q_0 + \vq) (q_0 \vv - \vv \times \vq)
       \\ &= -\inp{\vq}{q_0 \vv - \vv \times \vq} + q_0 (q_0 \vv - \vv \times \vq) + \vq \times {(q_0 \vv - \vv \times \vq)}
       \\ &= q_0^2 \vv - q_0 \vv \times \vq + q_0 \vq \times \vv - \vq \times {(\vv \times \vq)}
       \\ &= q_0^2 \vv - 2 q_0 \vv \times \vq - \vv \inp{\vq}{\vq} + \vq \inp{\vq}{\vv}
       \\ &= (\cos^2(\smallfrac{1}{2}\alpha) - \sin^2(\smallfrac{1}{2}\alpha)) v
           - 2 \cos(\smallfrac{1}{2}\alpha)\sin(\smallfrac{1}{2}\alpha) \, {v \times u}
       \\ &= \cos(\alpha) v + \sin(\alpha) \, {u \times v}
\end{align*}
This demonstrates that $q$ rotates $v$ anticlockwise by $\alpha$ radians about $u$.
We saw already that $x \mapsto q \cdot x$ is an orthogonal map with determinant $1$.
Therefore, $q$ maps to $\rho$.
\qed

\corollary[cor:transitive-s3-action]
$S^3$ acts transitively on $S^2\!$,
for every point on $S^2$ can be mapped
into any other point on $S^2$ by a rotation of the sphere.

The proof of theorem~\ref{thm:s3-surj-so3r} gives us a way to explicitly get
a $q \in S^3$ such that $q \cdot i = p$ for any $p \in S^2$:
we rotate $i$ onto $p$ with a rotation of $\R^3\!$.
If $p = -i$, $q = j$ will suffice,
so suppose $p \neq -i$.
Then an axis that we can rotate about is the one spanned by $i + p$,
which bisects the angle between $i$ and $p$,
so we need to rotate by $\pi$ radians.
We find
\begin{equationref}
\label{eqn:transitive-s3-action}
q = \frac{i + p}{\|i + p\|}
\end{equationref}
To verify that this works,
note that for $p \in S^2$ we have
$p \overline{p} = 1$ and $\overline{p} = -p$,
so $p^2 = -1$.
It then follows that
\[ p^2 = -1
% \implies pi + p^2 = -1 + pi
\enskip \implies \enskip p(i + p) = (i + p)i
\enskip \implies \enskip p(i + p)\overline{(i + p)} = (i + p)i\overline{(i + p)} \]
Multiplying by $\|i + p\|^{-2}$ on both sides then yields $p = q i q^{-1}$.

\subsection*{Physical interpretation}
Just like complex numbers are an extension of the real numbers,
quaternions are an extension of the complex numbers.
These extensions come at a cost:
when going from $\R$ to $\C$, you have to give up the ordering.
When going from $\C$ to $\H$, you have to give up commutativity.
Apart from their rich structure that is interesting in its own right,
quaternions have many useful applications.
By considering $S^3$ as a subset of $\H$,
it inherits a group structure.
Theorem~\ref{thm:s3-surj-so3r} tells us that this group is in a sense
twice $\SOR$: every rotation of $\R^3$ is represented by two antipodal quaternions.
When traversing a great circle through $1$ in $S^3\!$,
the points $1$ and $-1$ both correspond to the identity in $\SOR$.
This path in $S^3$ corresponds to a $4\pi$ rotation of $\R^3\!$,
and after a $2\pi$ rotation we will have moved from $1 \in S^3$ to $-1 \in S^3\!$.
This property is reminiscent of \emph{spinors},
and indeed proposition~\ref{prop:s3-isom-su2c} links the unit quaternions to the Pauli spin matrices.
