% !TEX root = ../thesis.tex

\null\vspace{5em}

\begin{center}
\textsc{\addfontfeature{LetterSpace = 10}abstract}
\end{center}

\begin{quote}
In this thesis we will investigate the Hopf map,
a differentiable map from the three-sphere to the two-sphere.
Its fibres, the inverse images of points on the sphere,
are circles that are all linked with every other fibre.
Based on the Hopf map we will construct divergenceless
vector fields that have a physical interpretation as
the magnetic field in the theory of magnetohydrodynamics.
The concept of linking relates to helicity in this theory,
a quantity that will be used to exhibit self-stable configurations of plasma.
\end{quote}

\clearpage

\null\vspace{5em}
\begin{center}
\begin{python}
from hopf import *
from math import pi
inner_points = [[-pi * 0.1, t * 2.0 * pi] for t in interval_open(8)]
outer_points = [[ pi * 0.1, t * 2.0 * pi] for t in interval_open(16)]
inner_fibres = [[projected_fibre_from_spherical(x), 'inner', 'back'] for x in inner_points]
outer_fibres = [[projected_fibre_from_spherical(x), 'outer', 'back'] for x in outer_points]
write_raw_draw_2d('generated/hatcher.tikz', 0.07, inner_fibres + outer_fibres)
\end{python}
\tikzexternalenable
\tikzsetnextfilename{hatcher}
\begin{tikzpicture}[scale = 2]
\definecolor{outerblue}{hsb}{0.6, 0.6, 0.5}
\definecolor{innerpurple}{hsb}{0.92, 0.8, 0.6}
\tikzstyle{back} = [line width = 4pt, white]
\tikzstyle{outer} = [outerblue, line width = 1pt]
\tikzstyle{inner} = [innerpurple, line width = 1pt]
\input{generated/hatcher.tikz}
\end{tikzpicture}
\end{center}

\vfill
\hfill\parbox{0.7\textwidth}{\raggedleft
\emph{Fibres of the Hopf map,
visualised through stereographic projection.
Inspired by the cover of \parencite{hatcher2002}.}}
\vspace{4.5em}
