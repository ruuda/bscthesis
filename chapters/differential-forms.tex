% !TEX root = ../thesis.tex

% Poincaré’s with better spacing, because the é clashes with the apostrophe.
\newcommand*{\poincares}{Poincaré\kern0.4pt ’\kern 0.1pt s\xspace}

\chapter{Differential forms}
\label{chap:differential-forms}
Up to now, we have investigated the Hopf map and its fibres.
Though interesting in its own right,
we eventually want to use this map to construct a magnetic field,
a vector field on $\R^3\!$.
Differential geometry gives us the tools to do so.
First, we will recall some of the terms involved.
Next, we will outline under which conditions several important vector spaces are isomorphic.
These isomorphisms will allow us to identify vector fields with differential forms.
Furthermore, we will apply this theory to the Hopf map,
and derive a vector field on $\R^3$ with various desirable properties.
Finally, we will define the \emph{Hopf invariant} of a differential form,
a quantity that will turn out to have an important physical interpretation.

\section{Manifolds and the exterior algebra}
For this chapter,
a little background in differential geometry is assumed.
Many concepts in differential geometry can be defined in various different
— but equivalent — ways.
Most applications in this chapter do not depend
on technical details of one particular definition.
Therefore we will mostly introduce notation here,
and we will restate a few definitions for convenience.
The definitions can all be found in chapter 1, 2 and 4 of \parencite{warner1971}.
Alternatively,
one may refer to section 6.4 and chapter 15 and 16 of \parencite{szekeres2004}.
In the following sections of this thesis,
we will assume \emph{differentiable} to mean $\Cinf\!$, i.e. infinitely differentiable.

\definition
A \emph{differentiable manifold of dimension $n$} is
a nonempty second countable Hausdorff space $M$
for which each point has a neighbourhood homeomorphic to an open subset of $\R^n\!$,
together with a \emph{differentiable structure} $\mathcal{A}$ of class $\Cinf\!$.
(Beware that \parencite{szekeres2004} does not require a manifold to be second countable.)
A pair $(U, \phi) \in \mathcal{A}$ of an open subset $U \subseteq M$,
and a continuous map $\phi : U \to \R^n$ that is a homeomorphism onto its image,
is called a \emph{coordinate chart}.
When there is no ambiguity,
we will refer to $M$ simply as a manifold.

\notation
Let $M$ be a manifold of dimension $n$, let $p \in M$.
The \emph{tangent space to $M$ at $p$} is written $\Tp M$.
It is a real vector space of dimension $n$.
The \emph{cotangent space to $M$ at $p$},
the dual space of $\Tp M$,
is written $\Tps M$.
A coordinate chart $(U, \phi)$ around $p_0 \in M$
induces a basis $(\ppart x_1, \ldots, \ppart x_n)$ on $\Tp M$ for all $p \in U$,
and thereby a dual basis $(dx_1, \ldots, dx_n)$ on $\Tps M$.

\definition
Let $M$ be a manifold.
Its \emph{tangent bundle} is defined as
\[ TM = \coprod_{p \in M} \Tp M \]
The tangent bundle comes with a natural projection map $\pi : TM \to M$
that sends a tangent vector $v \in \Tp M$ to $p$.

\definition
Let $V$ be an $n$-dimensional vector space over a field $F$ and $k \geq 0$ an integer.
The \emph{$k$-th exterior power of $V$}
is the unique (up to isomorphism)
vector space $\Lk V$ with a linear map $\wedge : V^k \to \Lk V$
such that,
for every alternating $k$-linear map
$f : V^k \to W$ to an $F$-vector space $W\!$,
there exists a unique $g : \Lk V \to W$ that makes the following diagram commute:
\vspace{-\parskip}
\begin{center}
\begin{tikzcd}[column sep = small] &
V^k   \ar[dl, swap, "\wedge"]
      \ar[dr, "f"] & \\
\Lk V \ar[rr, dashed, swap, "\exists_! g"] & & W
\end{tikzcd}
\end{center}
Elements of $\Lk V$ can be written as sums of
wedge products $v_1 \wedge \cdots \wedge v_k$
of $k$ elements of $V\!$,
and the map $\wedge$ is then given by $(v_1, \ldots, v_k) \mapsto v_1 \wedge \cdots \wedge v_k$.
It follows that $\L{1} V = V$.
By convention, $\L{0} V = F$.
One can show that $\dim(\Ln V) = 1$ and $\dim(\Lk V) = 0$ for all $k > n$.
$\Lk V$ is a subspace of \,$\bigwedge\!V\!$,
the \emph{exterior algebra} or \emph{Grassmann algebra} of $V\!$,
a graded-commutative $F$-algebra.

\notation
Let $M$ be a manifold and $k \geq 0$ an integer.
The \emph{space of differentiable $k$-forms}, written $\Omega^k M$,
is a subspace of the real vector space
\[ \prod_{p \in M} \Lk(\Tps M) \]
An element $\omega \in \Omega^k M$ is called a \emph{differentiable differential $k$-form}.
When there is no ambiguity,
we will refer to $\omega$ simply as a $k$-form.
$\Omega^k M$ is a subspace of $\Omega M$,
the \emph{exterior algebra} of $M$.
Elements of $\Omega^0 M$ may be identified with \emph{differentiable functions} $M \to \R$.

\definition
Let $M$ be a manifold of dimension $n$.
A \emph{vector field} on $M$ is a function $X : M \to TM$
that satisfies $\pi \circ X = \id_M$.
The vector field is called \emph{differentiable} if, \pagebreak
locally in a coordinate chart $(U, \phi)$,
it can be written as
\[ \sum_{i = 1}^{n} f_i \cdot \frac{\partial}{\partial x_i} \]
where the $f_i$ are differentiable functions $U \to \R$.
The set of differentiable vector fields on $M$ is a vector space,
denoted $\Upsilon M$.

\definition
Let $V$ be a real vector space of dimension $n$.
An \emph{orientation} on $V$ is an element of
\vspace{-0.5\parskip}
\[ \set{ \omega \in \Ln V \mid \omega \neq 0 } \modsim \]
where $\omega_1 \sim \omega_2$ if and only if $\omega_1 = \lambda \omega_2$
for some $\lambda \in \Rpos$.
Note that an ordered basis $(v_1, \ldots, v_n)$ for $V$ induces
an orientation $[v_1 \wedge \cdots \wedge v_n]$ on $V$.

\definition
Let $M$ be a manifold of dimension $n$.
An \emph{orientation} on $M$ is an element of
\vspace{-0.5\parskip}
\[ \set{ \omega \in \Omega^n M \mid \forall p \in M : \omega(p) \neq 0 } \modsim \]
where $\omega_1 \sim \omega_2$ if and only if $\omega_1 = f \cdot \omega_2$
for some differentiable function $f : M \to \R$.

\theorem[thm:exterior-algebra-functor]
Let $R$ be a commutative ring.
Taking the exterior algebra is a functor
from the category of $R$-modules
to the category of graded $R$-algebras.

\proof
See proposition~2 of \parencite[ch.~\textsc{iii}, \S~7.2]{bourbaki1970}.

\corollary[cor:exterior-power-functor]
Let $k$ be a nonnegative integer.
Then taking the $k$-th exterior power
is an endofunctor of the category of finite dimensional vector spaces over a field $F$.
If $V$ is a vector space over $F$ and $f : V \to V$ a linear endomorphism,
then theorem~\ref{thm:exterior-algebra-functor} yields
a unique $F$-algebra endomorphism of \,$\bigwedge\!V\!$,
which restricts to a linear map $\Lk V \to \Lk V$.
This map is given by the linear extension of
\[ v_1 \wedge \cdots \wedge v_k \longmapsto f(v_1) \wedge \cdots \wedge f(v_k) \]
See also equation~4 of \parencite[ch.~\textsc{iii}, \S~7.2]{bourbaki1970}.
For $k = 0$ the induced map is the identity map.

\section{Vector space isomorphisms}
For a general vector space,
there is little to say about its relation to other spaces.
When the vector space has additional properties however,
such as an inner product or an orientation,
several canonical identifications can be made.
These will play an important role in identifying
differential forms with differentiable functions
and vector fields,
which we will do at the end of this section.

\proposition[prop:isom-double-dual]
Let $V$ be a finite dimensional vector space over a field $F$.
Then $V$ is canonically isomorphic to its double-dual $V^{**}$ via the following linear isomorphism:
\[ \alpha : V \longto V^{**},
\quad v \longmapsto (u \mapsto u(v)) \]

\proof
It follows immediately that $u \mapsto u(v)$ is the zero function
if and only if $v = 0$.
Therefore, $\alpha$ is injective,
for its kernel is trivial.
Because $V$ is finite dimensional,
its dual space has the same dimension.
Consequently, $\alpha$ is surjective.
\qed

\definition[def:nondegenerate]
Let $V$ be a finite dimensional real vector space.
A bilinear form $B : V \times V \to \R$ is said to be \emph{nondegenerate}
if the map
\[ \chi : V \longto V^*, \quad v \longmapsto (u \mapsto B(v, u)) \]
is an isomorphism.

In general, a finite dimensional real vector space $V$
will not be canonically isomorphic to its dual space $V^*\!$.
A nondegenerate bilinear form (of which an inner product is an example)
establishes a canonical isomorphism between $V$ and $V^*\!$.
On a pseudo-Riemannian manifold,
every tangent space is equipped with a nondegenerate bilinear form.

\definition[def:pseudo-riemannian-manifold]
A \emph{pseudo-Riemannian manifold} is a pair $(M, B)$ of a manifold $M$
and a nondegenerate symmetric bilinear form $B_p : \Tp M \times \Tp M \to \R$
on every tangent space $\Tp M$ that satifies the following condition:
for all $X, Y$ differentiable vector fields on $M$,
the map
\vspace{-0.5\parskip}
\[ B(X, Y): M \longto \R, \quad p \longmapsto B_p(X(p), Y(p)) \]
is a differentiable function on $M$.

\lemma[lem:wedge-product-exists-nonzero]
Let $V$ be a real vector space of dimension $n$,
and let $k, l$ be integers such that $k + l = n$.
Then for all nonzero $\eta \in \Lk V$,
there exists an $\omega \in \Ll V$
such that $\eta \wedge \omega \neq 0$.

\proof
Let $v_1, \ldots, v_n$ be a basis for $V$
and suppose that $\eta \in \Lk V$ is nonzero.
It can be written as
\[ \eta = \!\sum_{1 \leq i_1 \leq \cdots \leq i_k \leq n}
            \lambda_{i_1 \cdots i_k} \, v_{i_1} \wedge \cdots \wedge v_{i_k}
\vspace{0.5\parskip} \]
with $\lambda_{i_1 \cdots i_k} \in \R$.
Because $\eta \neq 0$,
there exists a set of indices $i_1, \ldots, i_k$ such that $\lambda_{i_1 \cdots i_k} \neq 0$.
Let $j_1, \ldots, j_l$ be the subsequence of $(1, 2, \ldots, n)$ with $i_1, \cdots, i_k$ removed.
Define $\omega = v_{j_1} \wedge \cdots \wedge v_{j_l} \in \Ll V$.
In all terms of $\eta \wedge \omega$ except the one with coefficient $\lambda_{i_1 \cdots i_k}$,
at least one vector $v_i$ will occur twice in the wedge product
so these terms vanish.
Because $\lambda_{i_1 \cdots i_k} \neq 0$,
it follows that $\eta \wedge \omega \neq 0$.
\qed

\proposition[prop:isom-ext-codim-dual]
Let $V$ be a finite dimensional real vector space of dimension $n$,
and suppose that $\omega_0 \in \Ln V$ is given,
with $\omega_0 \neq 0$.
Then $\omega_0$ induces a linear isomorphism
$\L{n-1} V \to V^*\!$.

\proof
First, recall that $\Ln V$ has dimension $1$,
so every $\omega \in \Ln V$ can be written uniquely as
$\lambda \omega_0$ for some $\lambda \in \R$.
Thus, we have a linear isomorphism
\[ \rho : \Ln V \longto \R,
   \quad \lambda \omega_0 \longmapsto \lambda \]
This allows us to define the following linear map:
\[ \psi : \L{n-1} V \longto V^*,
\quad \omega \longmapsto (v \mapsto \rho(v \wedge \omega)) \]
Here $v \in V\!$.
Injectivity of $\psi$ follows from lemma~\ref{lem:wedge-product-exists-nonzero}.
Because $\dim V^* = \dim \L{n-1} V$,
$\psi$ is an isomorphism.
\qed

The form $\omega_0$ induces an orientation on $V\!$,
and an orientation on $V$ determines $\omega_0$ modulo a positive real factor.
When $V$ is equipped with a symmetric nondegenerate bilinear form $B$ as well,
$\omega_0$ can be fixed by requiring that it is a wedge product of an orthonormal basis for $V\!$.
An orientation on a manifold induces an orientation on its cotangent spaces.

\proposition[prop:unique-volume-element]
Let $V$ be a real oriented finite dimensional vector space
equipped with a symmetric nondegenerate bilinear form $B : V \times V \to \R$.
Then there exists a unique $\omega_0 = v_1 \wedge \cdots \wedge v_n \in \Ln V$
such that $(v_1, \ldots, v_n)$ is a positively oriented orthonormal basis for $V\!$.

\proof
First of all,
note that we can still talk about an orthonormal basis
even when $B$ is not positive-definite:
for a basis $(v_1, \ldots, v_n)$ for $V$ we require that
$|B(v_i, v_j)| = \delta_{i\!j}$.
With the Gram-Schmidt orthonormalisation process
it is always possible to construct an orthonormal basis for $V$.
See for instance theorem~5.2 of \parencite[p.~129]{szekeres2004}.
(Note that Szekeres does not require an inner product to be positive definite,
so his proof is applicable in our situation.)
If required, we reorder this basis to obtain a positively oriented orthonormal basis
$(v_1, \ldots, v_n)$ for $V\!$.
Finally, define $\omega_0 = v_1 \wedge \cdots \wedge v_n$.

Suppose that we have a different positively oriented orthonormal basis $(v_1', \ldots, v_n')$,
and $\omega_0' = v_1' \wedge \cdots \wedge v_n'$.
Then exists a linear isomorphism $f : V \to V$
that maps $(v_1, \ldots, v_n)$ onto $(v_1', \ldots, v_n')$.
Because both bases are positively oriented and orthonormal,
$\det(f) = 1$.
By corollary~\ref{cor:exterior-power-functor},
$f$ induces a unique linear map $\Ln f : \Ln V \to \Ln V$ that maps $\omega_0$ to $\omega_0'$.
Because $\Ln V$ has dimension $1$,
$\Ln f$ is given by $\omega \mapsto \lambda \omega$,
where $\lambda = \det(f)$.
(See definition 1 of \parencite[ch.~\textsc{iii}, \S~8.1]{bourbaki1970}.)
It follows that $f$ induces the identity on $\Ln V$,
so $\omega_0 = \omega_0'$.
\qed

\parencite[p.~218]{szekeres2004} provides an alternative proof
of proposition~\ref{prop:unique-volume-element} that uses Levi-Civita symbols
instead of universal constructions.

Putting together the isomorphisms earlier in this section
for a real vector space $V$ of dimension $n$,
we obtain the commutative diagram below.
The dashed arrows are induced by a nondegenerate bilinear form on $V\!$,
whereas the dotted arrows are induced by the choice of an element $\omega_0 \in \Ln(V^*)$.

\begin{center}
\begin{tikzcd}
V            \ar[r, "\chi", dashed]
             \ar[rr, "\alpha", bend left] &
V^*          \ar[r, dashed] &
V^{**}       \\
&
\L{n-1}(V^*) \ar[ul, dotted]
             \ar[ur, "\psi", swap, dotted] &
\end{tikzcd}
\end{center}

If $V$ is an oriented semi-inner product space
(a vector space with a symmetric nondegenerate bilinear form),
this establishes a canonical isomorphism between $\L{1}(V^*)$ and $\L{n-1}(V^*)$.

The above isomorphisms
can help to give an intuitive description of the various forms of the wedge product.
For a real $n$-dimensional vector space $V\!$,\,
we can make the following identifications:
\begin{itemize}
\item Taking the wedge product with $v \in \L{0} V$ is simply scalar multiplication.
\item If $V$ is an oriented semi-inner product space with bilinear form $B$,
      the wedge product corresponds to applying the bilinear form,
      in the sense that the following diagram commutes:
      \vspace{-\parskip}
      \begin{center}
      \begin{tikzcd}[row sep = large]
      V \times V                       \ar[r, "B"]
                                       \ar[d, "(\chi\textup{, } \psi^{-1} \circ \alpha)"] &
      \R \\
      \L{1} (V^*) \times \L{n-1} (V^*) \ar[r, "\wedge"] &
      \L{n} (V^*)                      \ar[u, swap, "\rho \circ \L{n} \chi^{-1}"]
      \end{tikzcd}
      \end{center}
\item If $V$ is an oriented semi-inner product space and $n = 3$,
      we have the following commutative diagram:
      \vspace{-\parskip}
      \begin{center}
      \begin{tikzcd}[row sep = large]
      V \times V                       \ar[r]
                                       \ar[d, "(\chi\textup{, } \chi)"] &
      V \\
      \L{1} (V^*) \times \L{1} (V^*) \ar[r, "\wedge"] &
      \L{2} (V^*)                      \ar[u, swap, "\alpha^{-1} \circ \psi"]
      \end{tikzcd}
      \end{center}
      For $V = \R^3$ with its standard orientation and inner product,
      the map $V \times V \to V$ is the cross product.
\end{itemize}

With the isomorphisms of vector spaces in this section,
we can identify spaces of differential forms
with the spaces of differentiable functions and vector fields.
For an $n$-dimensional manifold $M$,
we will need a nowhere-vanishing $\omega_0 \in \Omega^n M$ to make these identifications.
If $M$ is an oriented pseudo-Riemannian manifold,
there is one natural choice for $\omega_0$:
the \emph{volume form}.

\theorem[thm:vector-field-isom-one-form]
Let $(M, B)$ be a pseudo-Riemannian manifold of dimension $n$.
Then
\[ \phi : \Upsilon M \longto \Omega^1 M,
   \quad X \longmapsto B(X, {}\cdot{}) \]
is a linear isomorphism between the spaces of differentiable vector fields
and differentiable $1$-forms on $M$.

\proof
A coordinate chart on an open neighbourhood $U$ of $p_0 \in M$
induces a basis $(\partial / \partial x_1, \ldots, \partial / \partial x_n)$ on $\Tp M$
and $(dx_1, \ldots, dx_n)$ on $\Tps M$ for every $p \in U$.
We can express $B(X, {}\cdot{})$ locally as
\[ \sum_{i = 1}^n B(X, \partial / \partial x_i) \, dx_i \]
The vector field $p \mapsto \partial / \partial x_i$ defined on $U$ is differentiable for all $1 \leq i \leq n$,
so by definition~\ref{def:pseudo-riemannian-manifold},
$B(X, \partial / \partial x_i)$ is differentiable.
Therefore, $B(X, {}\cdot{})$ is a differentiable $1$-form.
Linearity of $\phi$ follows from bilinearity of $B_p$ for all $p \in M$.
Since by definition~\ref{def:nondegenerate} the map
$v \mapsto B_p(v, {}\cdot{})$ is a bijection from $\Tp M$ to $\Tps M$,
and $\L{1}(\Tps M) = \Tps M$,
we find that $\phi$ is an isomorphism.
\qed

\theorem[thm:unique-volume-form]
Let $(M, B)$ be an $n$-dimensional oriented pseudo-Riemannian manifold.
Then there exists a unique nowhere-vanishing $\omega_0 \in \Omega^n M$,
such that for all $p \in M$,
$\omega_0(p)$ is a wedge product of an orthonormal basis,
and $[\omega_0]$ is the orientation on $M$.
This $\omega_0$ is called the \emph{volume form}.

\proof
The pseudo-Riemannian structure of $M$
induces a nondegenerate symmetric bilinear form on $\Tp M$
for every $p \in M$.
By definition~\ref{def:nondegenerate} $\Tp M$ is isomorphic to $\Tps M$,
so we can transport the bilinear form to $\Tps M$.
The orientation on $M$ induces an orientation on $\Tps M$,
so proposition~\ref{prop:unique-volume-element} will give us
a unique nonzero $\xi_p \in \Ln(\Tps M)$.
This we can use to define
\[ \omega_0 : M \longto \coprod_{p \in M} \Ln(\Tps M),
   \quad p \longmapsto \xi_p \]
By construction $\omega_0$ is unique and nowhere-vanishing.
For every $p \in H$, $\omega_0(p)$ is a wedge product of an orthonormal basis for $\Tps M$,
and $[\omega_0]$ is the orientation on $M$.
The crux of this theorem however,
is that $\omega_0$ is differentiable.
This follows from the way in which an orthonormal basis for $\Tp M$ is constructed:
a coordinate chart on an open neighbourhood $U$ around $p_0 \in M$
induces a basis
$(\ppart x_1, \ldots, \ppart x_n)$ on $\Tp M$ for all $p \in U$,
so $p \mapsto \ppart x_1$ is a differentiable vector field on $U$.
By definition~\ref{def:pseudo-riemannian-manifold},
the function $\|\ppart x_1\| = |B(\ppart x_1, \ppart x_1)|$ is differentiable.
Next we apply the Gram-Schmidt process,
but with differentiable vector fields on $U$.
This process involves taking sums of differentiable vector fields
and dividing by their norm.
If $B$ is not positive-definite,
the norm could vanish even if the vector field does not,
but by shrinking $U$ if necessary,
we can ensure that the norm does not vanish on $U$,
so the process results in differentiable vector fields.
Theorem~\ref{thm:vector-field-isom-one-form} then yields
a set of $n$ differentiable $1$-forms $\eta_i \in \Omega^1 M$,
$1 \leq i \leq n$,
such that $(\eta_1(p), \ldots, \eta_n(p))$ is an orthonormal basis for $\Tps M$ at every $p \in U$.
We can write $\omega_0 = \eta_1 \wedge \cdots \wedge \eta_n$,
so it follows that $\omega_0$ is a differentiable $n$-form.
\qed
% Note: see also page 149 of Warner.

\theorem[thm:vector-field-isom-codim-form]
Let $M$ be a manifold of dimension $n$,
and let $\omega_0 \in \Omega^n M$ be a nowhere-vanishing $n$-form.
(From theorem~\ref{thm:unique-volume-form},
it follows that an oriented pseudo-Riemannian manifold
is naturally equipped with such a form.)
Then $\omega_0$ induces a natural linear isomorphism
\vspace{-0.5\parskip}
\[ \phi : \Omega^{n-1} M \longto \Upsilon M \vspace{0.5\parskip} \]

\vspace{-\parskip}
\proof
Let $\eta \in \Omega^{n-1} M$ be given.
By proposition~\ref{prop:isom-ext-codim-dual}
and definition~\ref{def:nondegenerate}
we have an isomorphism $\phi_p : \L{n-1}(\Tps M) \to \Tp M$ for every $p \in M$,
so we can define $\phi(\eta) = p \mapsto \phi_p(\eta(p))$.
Say $\phi(\eta) = X$, then we must show that $X$ is differentiable.
It suffices to verify this locally in an open neighbourhood $U$ around $p_0 \in M$.
A coordinate chart on $U$ induces a basis $(\ppart x_1, \ldots, \ppart x_n)$ on $\Tp M$
and a basis $(dx_1, \ldots, dx_n)$ on $\Tps M$ for all $p \in U$,
so $p \mapsto dx_i$ is a differentiable one-form on $U$ for all $1 \leq i \leq n$.
We can then write $dx_i \wedge \eta$ as $f_i \cdot \omega_0$
for differentiable functions $f_i : U \to \R$,
because $\omega_0$ is nowhere-vanishing.
The vector field $X : M \to TM$ is then locally given by
\[ X = \sum_{i = 1}^n f_i \cdot \frac{\partial}{\partial x_i} \]
It follows that $X$ is differentiable.
Because every $\phi_p$ is a linear isomorphism,
$\phi$ is itself a linear isomorphism.
\qed

\example[ex:r3-manifold]
For all positive integers $n$,
$\R^n$ is a differentiable manifold of dimension $n$.
If we take the standard orientation and Euclidean inner product,
then $\R^n$ is an oriented Riemannian manifold.
Cartesian coordinates induce a basis
$(\ppart x_1, \ldots, \ppart x_n)$ on every tangent space,
which we can identify with the standard basis $(e_1, \ldots, e_n)$.
The isomorphism from theorem~\ref{thm:vector-field-isom-one-form}
maps $(\ppart x_1, \ldots, \ppart x_n)$ to a basis for the cotangent space,
which for the standard inner product on $\R^n$
coincides with the dual basis $(dx_1, \ldots, dx_n)$.
Theorem~\ref{thm:vector-field-isom-codim-form} tells us that we can identify
elements of $\Omega^{n-1} \R^n$ with vector fields via the identification
\[ dx_1 \wedge \cdots \wedge dx_{i - 1} \wedge dx_{i + 1} \wedge \cdots \wedge dx_n
\longmapsto (-1)^{i + 1} e_i \]
The wedge product of all basis vectors except $dx_i$ corresponds to $e_i$.
Furthermore, via the volume form $dx_1 \wedge \cdots \wedge dx_n$,
we can identify an $n$-form $f \cdot dx_1 \wedge \cdots \wedge dx_n$
with the differentiable function $f$ on $\R^n\!$.
To summarise:
in $\R^n\!$, we can identify one-forms and $(n-1)$-forms with differentiable vector fields,
and we can identify zero-forms and $n$-forms with differentiable functions.

\section{Constructing a vector field}
\label{sec:constructing-a-vector-field}
With the tools of the previous section
we can identify vector fields and $k$-forms on manifolds.
In particular,
we are interested in a divergenceless vector field on $\R^3$
that we will interpret as a magnetic field in section~\ref{sec:constructing-a-magnetic-field}.
The operator that allows us to express divergence
is the \emph{exterior derivative} that we will introduce in this section.
Furthermore, we will recall how differentiable functions
between manifolds introduce a \emph{pullback} on their exterior algebras.
Finally, we will combine the two to construct a divergenceless vector field.

\notation[not:differential]
Let $M$ be a manifold and $f \in \Omega^0 M$ a differentiable function.
Then its \emph{differential},
written $\df$,
is an element of $\Omega^1 M$.
It is a differentiable function $TM \to \R$
that is linear on every tangent space.
For a definition one may refer to definition~1.22 of \parencite[p.~16]{warner1971}
or \parencite[p.~423]{szekeres2004}.

For a manifold $M$ of dimension $n$ and $f \in \Omega^0 M$,
locally in a coordinate chart $(U, \phi)$
the differential $\df$ can be expressed as
\begin{equationref}
\label{eqn:differential}
\df = \sum_{i = 1}^n \frac{\partial (\phi^{-1} \circ f)}{\partial x_i} \, dx_i
\end{equationref}

Note that this notation is compatible with the usage of $dx_i$
as a cotangent vector: in a coordinate chart $(U, \phi)$
the function $x_i : U \to \R$ that sends a point $p$ to the
$i$-th coordinate of $\phi(p)$ has differential $dx_i$.

\theorem
Let $M$ be a manifold.
Then for each integer $k \geq 0$,
there exists a unique linear operator
$d : \Omega^k M \to \Omega^{k+1} M$,
called the \emph{exterior derivative},
that satisfies the following properties:
\begin{itemize}
\item For $f \in \Omega^0 M$, $\df$ is the differential of $f$.
\item $d^2 = 0$.
\item For $\omega \in \Omega^i M$ and $\eta \in \Omega^j M$
      it holds that
      $ d(\omega \wedge \eta) = d\omega \wedge \eta + (-1)^i \omega \wedge d\eta$.
      An operator that satisfies this property is called an \emph{anti-derivation}.
\end{itemize}

\proof
See theorem~2.20 of \parencite[p.~65]{warner1971} or section~16.1 of \parencite[p.~448]{szekeres2004}.

The anti-derivation property of $d$ is the analogue of the product rule for differentiation.
It allows us to reduce computations to the case of equation~\ref{eqn:differential}.
A $k$-form $\omega \in \Omega^k M$ is said to be \emph{closed} if $d\omega = 0$.
It is said to be \emph{exact} if there exists an $\alpha \in \Omega^{k - 1} M$ such that $d\alpha = \omega$.

\theorem{\poincares lemma}[thm:poincare]
Let $M$ be a contractible manifold, $k > 1$ an integer.
Then every closed $k$-form on $M$ is exact.

\proof
See for instance corollary \textsc{a} of \parencite[p.~156]{warner1971}
(although its preconditions are slightly different)
or corollary~4.1.2.1 of \parencite[p.~36]{bott1982}.

\poincares lemma is in fact a corollary of a more general result involving de Rham cohomology.
We will not go in depth here,
although there are some interesting consequences for $S^3\!$.
We highlight one particular result:

\proposition[prop:poincare-s3]
Let $\omega \in \Omega^2 S^3$ be closed.
Then $\omega$ is exact.

\proof
This follows from the fact that $H^2(S^3) = 0$,
where $H^2(S^3)$ is the second de Rham cohomology group,
defined as
\[ H^2(S^3) = \frac{\ker(d: \Omega^2 S^3 \to \Omega^3 S^3)}{\im(d: \Omega^1 S^3 \to \Omega^2 S^3)} \]
See also \parencite[p.~36]{bott1982}.
\qed

\example[ex:grad-curl-div]
For $\R^3$ with standard orientation and inner product,
the $d$ operator has well-known interpretations.
By \poincares lemma the following sequence of vector spaces is exact:
\begin{center}
\begin{tikzcd}[column sep = 2em]
0             \ar[r] &
\R            \ar[r] &
\Omega^0 \R^3 \ar[r, "d_0"] &
\Omega^1 \R^3 \ar[r, "d_1"] &
\Omega^2 \R^3 \ar[r, "d_2"] &
\Omega^3 \R^3 \ar[r] &
0
\end{tikzcd}
\end{center}
As we saw in the previous section,
we can identify vector fields with $1$-forms and $2$-forms
and differentiable functions with $0$-forms and $3$-forms on $\R^3\!$.
Under these identifications,
we have the following correspondences:
\begin{itemize}
\item $d_0$ corresponds to taking the gradient of a function.
\item $d_1$ corresponds to taking the curl of a vector field.
\item $d_2$ corresponds to taking the divergence of a vector field.
\end{itemize}
Let $f : \R^3 \to \R$ be a differentiable function
and $F : \R^3 \to \R^3$ a differentiable vector field.
Because $d^2 = 0$,
we get the following identities for free:
\[ \nabla \times {(\nabla f)} = 0
   \qquad\qquad
   \nabla \cdot {(\nabla \times F)} = 0 \]
More generally,
in $\R^n$ with standard orientation and inner product,
the operator $d_0$ corresponds to the gradient and $d_{n-1}$ corresponds to the divergence.

Apart from the exterior derivative,
we will need another concept for the construction of a divergenceless vector field:
the pullback.
Recall that an linear map $f : V \to W$ between vector spaces
induces a dual map (sometimes called transpose map)
$f^* : W^* \to V^*$ between the dual spaces.
Similarly, a function between tangent bundles of manifolds induces a dual map,
and by functorality of the exterior algebra,
this allows us to \emph{pull back} $k$-forms.
Definitions of the concepts below
can be found in chapter~1 of \parencite[p.~16]{warner1971}
and section~15.4 of \parencite[p.~426]{szekeres2004}.

\notation
Let $M, N$ be manifolds and $f : M \to N$ a differentiable function.
For every $p \in M$, $f$ induces a \emph{tangent map} in $p$,
a linear map
\[ f_*(p) : \Tp M \longto T_{\!f(p)} N \]
The tangent map of $f$ is sometimes called the \emph{differential} of $f\!$,
and in fact the differential $d\hspace{-1pt}g$ of a differentiable function $g : M \to \R$
as introduced in notation~\ref{not:differential} is the tangent map $g_*$
when $\Tp\R$ is identified with $\R$ for every $p \in \R$.

\definition
Let $M, N$ be manifolds and $f : M \to N$ a differentiable function.
By theorem~\ref{thm:exterior-algebra-functor},
the dual maps of the tangent maps induce a map
\[ f^* : \Omega N \longto \Omega M \]
As shown in proposition~2.23\textsc{a} of \parencite[p.~68]{warner1971},
$f^*$ is an algebra homomorphism.
For $\omega \in \Omega^k N$,
the element $f^*(\omega) \in \Omega^k M$ is called
the \emph{pullback} of $\omega$ by $f$.

\theorem
The pullback commutes with the exterior derivative.
In other words,
for manifolds $M, N$ and a differentiable function $f: M \to N$
it holds that for all $\omega \in \Omega N$ we have $f^*(d\omega) = d(f^*(\omega))$.

\proof
See proposition~2.23\textsc{b} of \parencite[p.~68]{warner1971}
or theorem~16.2 of \parencite[p.~451]{szekeres2004}.

Commutativity of the pullback and exterior derivative is what allows
us to construct divergenceless vector fields.
Suppose that we have $\xi \in \Omega^n M$
on an $n$-dimensional manifold $M$.
Then $\Omega^{n + 1} M = 0$,
so $d\xi = 0$.
If we now take a different manifold $N$ of larger dimension,
then $\Omega^{n + 1} N \neq 0$.
If we have a differentiable function $f : N \to M$,
then $d(f^*(\xi)) = f^*(d\xi) = 0$;
the $n$-form $f^*(\xi)$ is closed.
We will apply this procedure to the Hopf map and the stereographic projection,
but first we have to verify some technicalities.
First of all, we need a nowhere-vanishing two-form on $S^2\!$.
We have not equipped $S^2$ with a pseudo-Riemannian structure,
so we cannot apply theorem~\ref{thm:unique-volume-form} here.
While we could embed the tangent spaces of $S^2$ in $\R^3$
and restrict the Euclidean inner product of $\R^3$ to the tangent spaces,
we will take a different route that has the same result.
There is one preferred choice of two-form:
because $S^2$ is invariant under rotation,
we require the two-form to be $\SOR$-invariant.

\lemma[lem:invariant-two-form]
There exists a nowhere-vanishing two-form $\omega_0 \in \Omega^2 S^2$
which is invariant under $\SOR$.

\proof
Define in $\R^3$ the two-form
\[ \omega = x_1 \cdot dx_2 \wedge dx_3
          \ + \ x_2 \cdot dx_3 \wedge dx_1
          \ + \ x_3 \cdot dx_1 \wedge dx_2 \]
Then define $\omega_0 = i^*(\omega)$,
where $i : S^2 \inj \R^3$ is the inclusion.
Under the isomorphism $\Omega^2 \R^3 \to \Upsilon \R^3\!$
from theorem~\ref{thm:vector-field-isom-codim-form},
$\omega$ corresponds to a vector field pointing radially outward,
the identity function on $\R^3$ in fact.
This field is $\SOR$-invariant.
Because elements of $\SOR$ are orientation-preserving
and orthogonal, it follows that $\omega$ and thereby $\omega_0$
are invariant under $\SOR$.
Furthermore, because $\omega_0$ does not vanish on e.g.
$(0, 1, 0)$,
it follows from invariance that it vanishes nowhere.
\qed

\lemma
The inverse stereographic projection $\pi^{-1} : \R^3 \to S^3$
and the Hopf map $h : S^3 \surj S^2$ are differentiable maps.

\proof
As we can see in equation~\ref{eqn:inverse-stereographic-projection} and \ref{eqn:hopf-coordinates},
$\pi^{-1}$ is given by a rational function of polynomials on every coordinate
(with a denominator that is positive everywhere),
and $h$ is given by a polynomial expression on every coordinate.
It follows that both maps are differentiable.
\qed

\example
Consider $\R^3\!$, $S^3\!$, and $S^2\!$.
We have the maps $h : S^3 \to S^2$ and $\phi = h \circ \pi^{-1} : \R^3 \to S^2\!$.
To a differentiable function $f : S^2 \to \R$
we can assign the two-form $\xi = f \cdot \omega_0$
with $\omega_0$ as in lemma~\ref{lem:invariant-two-form}.
Because $S^2$ has dimension two, we have $d\xi = 0$.
By pulling back $\xi$ we obtain the closed two-forms
\[ \beta = \phi^*(\xi) \in \Omega^2 \R^3, \quad d\beta = 0
   \qquad\textup{and}\qquad
   \gamma = h^*(\xi) \in \Omega^2 S^3, \quad d\gamma = 0 \]
As we saw before,
we can identify two-forms on $\R^3$ with vector fields,
and the exterior derivative then corresponds to taking the divergence.
We can identify $\beta \in \Omega^2 \R^3$ with $B : \R^3 \to \R^3\!$.
Closedness, $d\beta = 0$, then translates to $\nabla \cdot B = 0$.
Hence, we have constructed a divergenceless vector field on $\R^3$
from a differentiable function $f : S^2 \to \R$.
The function $\phi$ is constant on field lines of $B$.

We can even say more about $\beta$ and $\gamma$.
By \poincares lemma (theorem~\ref{thm:poincare}),
there exists an $\alpha \in \Omega^1 \R^3$ such that $d\alpha = \beta$.
Translated to vector fields,
this means that there exists a vector field $A : \R^3 \to \R^3$
such that $\nabla \times A = B$.
In physics, this field is called a \emph{vector potential},
and it will play an important role in chapter~\ref{chap:magnetohydrodynamics}.
Moreover, by proposition~\ref{prop:poincare-s3} \linebreak $\alpha$ is the pullback
of a one-form on $S^3\!$,
an analogue of a vector potential on the three-sphere.

For an explicit computation of the field induced on $\R^3$
by $\omega_0 \in \Omega^2 S^2\!$,
recall that we have a sequence of differentiable maps:
\begin{center}
\begin{tikzcd}
\R^3 \ar[r, hook, "\pi^{-1}"] &
S^3  \ar[r, two heads, "h"] &
S^2  \ar[r, hook, "i"] &
\R^3
\end{tikzcd}
\end{center}
We define the function $\phi: \R^3 \to \R^3$ to be the composition of these maps.
As the composition of differentiable functions
$\phi$ is differentiable,
so we can compute the pullback by $\phi$ of $\omega$
as defined in lemma~\ref{lem:invariant-two-form}.
This will be the pullback of $\omega_0$ by $h \circ \pi^{-1}$.
Using equation~\ref{eqn:inverse-stereographic-projection}
and \ref{eqn:hopf-coordinates}
we can express $\phi$ as
\begin{equationref}
    \phi(x_1, x_2, x_3)
% = h\left( \frac{1}{\nsq{x} + 1} \left(\nsq{x} - 1, \, 2x_1, 2x_2, 2x_3 \right) \right)
% = \frac{1}{(\nsq{x} + 1)^2}
%   \begin{pmatrix}
%   (\nsq{x} - 1)^2 + 4x_1^2 - 4x_2^2 - 4x_3^2 \\
%   2(4x_1 x_2 - 2(\nsq{x} - 1)x_3) \\
%   2(2(\nsq{x} - 1)x_2 + 4x_1 x_3)
%   \end{pmatrix}
  = \frac{4}{(\nsq{x} + 1)^2}
    \begin{pmatrix}
      \smallfrac{1}{4}(\nsq{x} - 1)^2 - \nsq{x} + 2x_1^2 \\
      2x_1 x_2 - (\nsq{x} - 1)x_3 \\
      2x_1 x_3 + (\nsq{x} - 1)x_2
    \end{pmatrix}
\end{equationref}
Here $\nsq{x} = x_1^2 + x_2^2 + x_3^2$.
Before we compute the pullback of $\omega$ by $\phi$,
we will compute the pullback by a general differentiable function $f : \R^3 \to \R^3$
with component functions $f_1, f_2, f_3$.
Observe that for $i = 1, 2, 3$,
\[    f^*(dx_i)
\ = \ d(f^*(x_i))
\ = \ d(x_i \circ f)
\ = \ \df_i \]
The pullback of $\omega$ by $f$ is given by
\begin{equationref}
 f^*(\omega)
 = \sum_{(i,j,k)} \left(
   \sum_{(p,q,r)} f_p \cdot \left(
       \frac{\partial f_q}{\partial x_j} \frac{\partial f_r}{\partial x_k}
     - \frac{\partial f_q}{\partial x_k} \frac{\partial f_r}{\partial x_j}
   \right) \right) dx_j \wedge dx_k
\end{equationref}
where $(i, j, k)$ and $(p, q, r)$ run over $\set{(1, 2, 3), (2, 3, 1), (3, 1, 2)}$.
Via the isomorphisms in the previous section we
obtain a vector field, its $i$-th coordinate given by the term $dx_j \wedge dx_k$.
For $f = \phi$,
we find the field
\begin{equationref}
\label{eqn:hopf-field}
 H(x_1, x_2, x_3)
 = \frac{16}{(\nsq{x} + 1)^3}
   \begin{pmatrix}
   1 + x_1^2 - x_2^2 - x_3^2 \\
   2(x_1 x_2 + x_3) \\
   2(x_1 x_3 - x_2)
   \end{pmatrix}
\end{equationref}
This field coincides with the field in \parencite[p.~31]{dalhuisen2014}
exept for a factor $-1$ and a point reflection $x \mapsto -x$.
(Such a point reflection is called a \emph{parity transformation} in physics.)
This is because the projection point of the stereographic projection differs.

Now that we have an expression for the field,
consider a general two-form $\xi \in \Omega^2 S^2\!$.
It can be written as $f \cdot \omega_0$
for some differentiable function $f : S^2 \to \R$,
so for the pullback it holds that
\[ \phi^*(\xi)
 \ = \ \phi^*(f \cdot \omega_0)
 \ = \ (f \circ \phi) \cdot \phi^*(\omega_0) \]
Consequently,
the pullback of $\xi$ will correspond to the product
of the vector field $H$ with the real-valued function $f \circ \phi$.

We can summarise the main result of this section with the following statement:

\begin{quote}
Given a differentiable function $\phi : \R^3 \to M$,
where $M$ is an oriented pseudo-Riemannian manifold of dimension two,
and a differentiable function $f : M \to \R$,
we can construct a divergenceless vector field on $\R^3\!$.
The function $\phi$ determines the structure of the field;
$\phi$ is constant along field lines.
The function $f$ determines the magnitude of the field;
multiplying $f$ by a differentiable function $g : M \to \R$
will change the field magnitude by a factor $g \circ \phi$.
\end{quote}

The mathematics in this chapter do admit a physical interpretation,
but rather than giving a brief physical summary here,
we will defer the physics to the next chapter,
section~\ref{sec:constructing-a-magnetic-field}.

\section{The Hopf invariant}
\label{sec:the-hopf-invariant}
We have seen before that the fibres of the Hopf map are linked,
and that we can construct a vector field with linked field lines.
In order to quantify the amount of linking in a vector field,
we will define the \emph{Hopf invariant},
a quantity that is invariant under orientation-preserving diffeomorphisms.
Although its definition is purely algebraic,
it has an important physical interpretation as the \emph{helicity} of a field.
This connection and the relation to fluid flow will be explored briefly.
In section~\ref{sec:linked-and-knotted-fields} we will make the connection to magnetohydrodynamics.

The definition of the Hopf invariant given below follows \parencite{arnold1974},
because it most closely aligns with its applications in chapter~\ref{chap:magnetohydrodynamics}.
It differs from the more common definition of the Hopf invariant
(given in \parencite[p.~228]{bott1982}, for example),
in the sense that we define the Hopf invariant as a property of a two-form,
not as a property of a map between spheres.
Nevertheless, both definitions involve the same integral.

\definition[def:hopf-invariant]
Let $M$ be an oriented three-dimensional manifold.
Let $\xi \in \Omega^2 M$ be a compactly supported two-form that is exact,
i.e. there exists an $\alpha \in \Omega^1 M$ such that $d\alpha = \xi$.
Then the \emph{Hopf invariant} of $\xi$ is defined to be
\[ H(\xi) = \int_M \alpha \wedge d\alpha \]
To show that this definition is independent of the choice of $\alpha$,
suppose that $\beta \in \Omega^1 M$ also satisfies $d\beta = \xi$.
Then we have
\begin{align*}
      \int_M \alpha \wedge d\alpha \ - \int_M \beta \wedge d\beta
%  &= \int_M (\alpha \wedge d\alpha) - (\beta \wedge d\alpha)
   &= \int_M (\alpha - \beta) \wedge d\alpha
\intertext{Because $d((\alpha - \beta) \wedge \alpha)
           = d(\alpha - \beta) \wedge \alpha - (\alpha - \beta) \wedge d\alpha$, we find}
   &= \int_M d(\alpha - \beta) \wedge \alpha - d((\alpha - \beta) \wedge \alpha)
\intertext{The first term vanishes because $d(\alpha - \beta) = \xi - \xi = 0$,
so we are left with}
   &= \int_M d((\beta - \alpha) \wedge \alpha)
\end{align*}
Using Stokes’ theorem,
we can write this as an integral over $\partial M$
which vanishes because $\partial M = \emptyset$.
(See also \parencite[p.~148]{warner1971}.)
Therefore, the Hopf invariant is well-defined.

The requirement that $\xi$ has compact support
prevents us from defining the Hopf invariant for a nowhere-vanishing two-form on $\R^3\!$.
If the two-form happens to be the pullback of a two-form on $S^3\!$
as discussed in the previous section,
we can compute the Hopf invariant on $S^3$ instead.
For subsets of a manifold $M$ we can also define a Hopf invariant,
but the argument that $\partial M = \emptyset$ cannot be used anymore
to ensure that such an invariant is well-defined.
To resolve this issue,
we must require that the subset is well-behaved with respect to $\xi$.

\definition
Let $M$ be a manifold of dimension $n$ and $D \subseteq M$.
$D$ is called a \emph{regular domain} if for every $p \in M$
one of the following conditions holds:
\begin{itemize}
\item There exists an open neigbourhood of $p$ contained in $M \setminus D$.
\item There exists an open neighbourhood of $p$ contained in $D$.
\item There exists a centered chart $(U, \phi)$ about $p$
      such that $\phi(U \cap D) = \phi(U) \cap H^n$
      with $H^n = \set{x \in \R^n \mid x_0 \geq 0 }$.
\end{itemize}
The domain $D$ is said to have a \emph{smooth boundary} $\partial D$.
Regular domains are subsets of manifolds that we can integrate $n$-forms over
and where we can apply Stokes’ theorem.
(See also \parencite[p.~145]{warner1971}.)

\definition[def:compatible-subset]
Let $M$ be a three-dimensional manifold.
Let $U \subseteq M$ be open in $M$ with smooth boundary $\partial U$,
such that its closure $\overline{U}$ in $M$ is compact.
Denote by $i : \partial U \inj M$ the inclusion.
Let $\xi \in \Omega^2 M$ be an exact two-form.
$U$ is said to be \emph{compatible with $\xi$} if $i^*(\xi) = 0$.

In $\R^3\!$, where $\xi$ can be identified with a vector field,
this definition has a geometric interpretation:
$U$ is compatible with $\xi$ if $\xi$ is tangent to $\partial U$.

\definition[def:hopf-invariant-of-subset]
Let $M$ be an oriented three-dimensional contractible manifold.
Let $\xi \in \Omega^2 M$ be an exact two-form,
i.e. there exists an $\alpha \in \Omega^1 M$ such that $d\alpha = \xi$.
Let $U \subseteq M$ be open in $M$ with smooth boundary $\partial U$,
such that its closure $\overline{U}$ in $M$ is compact,
and such that $U$ is compatible with $\xi$.
Then the \emph{Hopf invariant of $\xi$ restricted to $U$} is defined to be
\vspace{-0.3em}
\[ H(\xi\,|_U) = \int_U \alpha \wedge d\alpha \vspace{0.3em} \]
Again, we have to show that this definition does not depend on the choice of $\alpha$.
Suppose that $\beta \in \Omega^1 M$ satisfies $d\beta = \xi$,
then $\alpha - \beta$ is closed.
Because $M$ is contractible, it follows from \poincares lemma (theorem~\ref{thm:poincare})
that $\beta$ can be written as $\alpha + \df$ for some $f \in \Omega^0 M$.
Exploiting this, we find
\begin{align*}
   \int_U \alpha \wedge d\alpha \ - \int_U \beta \wedge d\beta
   &= \int_U \df \wedge \xi
\intertext{Because $d(f \cdot \xi) = \df \wedge \xi + f \cdot d\xi$ we can write this as}
   &= \int_U d(f \cdot \xi) - f \cdot d \xi
\intertext{The factor $d\xi$ vanishes because $d\xi = d^2 \alpha = 0$,
so we may apply Stokes’ theorem}
  &= \int_{\partial U} (f \circ i) \cdot i^*(\xi)
\end{align*}
Here $i : \partial U \inj M$ denotes the inclusion.
Because $\xi$ is compatible with $U$,
$i^*(\xi) = 0$,
so the integral evaluates to zero.
Therefore, the Hopf invariant restricted to $U$ is well-defined.

\definition[def:orientation-preserving]
Let $M$ be an oriented $n$-dimensional manifold
with orientation $[\omega_0]$ for some nowhere-vanishing $\omega_0 \in \Omega^n M$.
Then a diffeomorphism $f : M \to M$ is said to be \emph{orientation-preserving}
if $[f^*(\omega_0)] = [\omega_0]$.
This does not depend on the choice of $\omega_0$,
for if $\omega_0 = g \cdot \omega_0'$
for a differentiable function $g : M \to \R$ and $\omega_0' \in \Omega^n M$ nowhere-vanishing,
then $f^*(\omega_0) = (g \circ f) \cdot f^*(\omega_0')$.
Because $f$ is a diffeomorphism,
$f^*(\omega_0)$ will be nowhere-vanishing,
so $[f^*(\omega_0)]$ is well-defined.

\proposition[prop:integral-is-invariant]
Let $M$ be an oriented $n$-dimensional manifold
and let $f : M \to M$ be an orientation-preserving diffeomorphism.
Suppose $U \subseteq M$ has a smooth boundary.
Let $\xi \in \Omega^n M$ be such that $\Supp(\xi) \cap \overline{f(U)}$ is compact.
Then it holds that
\[ \int_U \, f^*(\xi) = \int_{f(U)} \xi \]

\vspace{-\parskip}
\proof
See \parencite[p.~148]{warner1971}.

From this proposition it follows that the Hopf invariant is actually an invariant:
it is invariant under orientation-preserving diffeomorphisms of $M$.

\corollary[cor:hopf-invariant-is-invariant]
Let $M$ and $\xi$ be as in definition~\ref{def:hopf-invariant},
and let $f : M \to M$ be an orientation-preserving diffeomorphism.
Then it holds that
\[ H(\xi) = H(f^*(\xi)) \]

\corollary[cor:hopf-invariant-of-subset-is-invariant]
Let $M$, $\xi$ and $U$ be as in definition~\ref{def:hopf-invariant-of-subset}.
Let $f : M \to M$ be a orientation-preserving diffeomorphism.
Then it holds that
\[ H(f^*(\xi)|_U) = H(\xi\,|_{f(U)}) \]

\vspace{-\parskip}
So far, we have shown that the Hopf invariant is indeed an invariant,
but we have yet to show how it relates to linking.
When $\xi$ corresponds to a vector field,
we can give a heuristic argument involving involving flux tubes,
which we will do in section~\ref{sec:linked-and-knotted-fields}.
A more formal relation between the Hopf invariant and linking number
is established in \parencite{arnold1974},
and the correspondence with several alternative definitions of linking number
is given in \parencite[pp.~229–234]{bott1982}.

\subsection*{Physical interpretation}
The formalism in this section admits an almost 1\,:\,1 translation
to a physical situation.
The exact two-form will be replaced by the divergenceless vector field $\Bf$,
and \emph{helicity} will play the role of the Hopf invariant.

In definition~\ref{def:hopf-invariant} and \ref{def:hopf-invariant-of-subset}
we defined the Hopf invariant of an exact two-form $\xi$.
Exactness means that there exists a one-form $\alpha$ such that $d\alpha = \xi$,
and because $d^2 = 0$ this implies that $d\xi = 0$.
As we saw in example~\ref{ex:r3-manifold} and \ref{ex:grad-curl-div},
in $\R^3$ we can identify both one-forms and two-forms with vector fields,
and the $d$ operator corresponds to the curl and divergence.
The two-form $\xi$ would correspond to a vector field $\Bf$ with $\nabla \cdot \Bf = 0$,
and $\alpha$ would correspond to a vector potential $\Af$ with $\nabla \times \Af = \Bf$.
In this case the wedge product corresponds to the dot product,
and the Hopf invariant of $\Bf$ restricted to a volume $U \subseteq \R^3$
can be expressed as the \emph{helicity} $H$ of $\Bf$ in $U$:
\[ H \, = \iiint_{\!U} \Af \cdot \Bf \ dx^3 \]
The choice of vector potential is not unique.
We may apply a gauge transformation $\Af \to \Af + \nabla \cdot f$
for some differentiable function $f : \R^3 \to \R$,
and this will satisfy $\nabla \times {(\Af + \nabla \cdot f)} = \Bf$ as well.
It is not obvious that helicity is gauge invariant,
and in fact it only is if $U$ satisfies some requirements.
The statement that $U$ is compatible with $\xi$
as defined in definition~\ref{def:compatible-subset},
corresponds in $\R^3$ with the statment that $\Bf$ is tangent to the boundary of $U$ everywhere.
We may take $U$ to be a bounded \emph{flux tube},
a volume enclosed by a surface of field lines of $\Bf$,
such that no field line penetrates the boundary of $U$.
The check of well-definedness of the Hopf invariant in definition~\ref{def:hopf-invariant-of-subset}
can be translated directly into a proof of the gauge invariance of the helicity.
In the context of electrodynamics,
the application of Stokes’ theorem in the proof
is often called \emph{Gauss’ law}.

In corollary~\ref{cor:hopf-invariant-of-subset-is-invariant}
we have shown that the Hopf invariant
is invariant under an orientation-preserving diffeomorphism.
An orientation-preserving diffeomorphism of $\R^3$
is a differentiable function $f : \R^3 \to \R^3$ that does not invert parity.
An example of such a function
can be constructed from a fluid flow.
If the position of a test particle as a function of time is given by $\rf(t)$,
then the map $\rf(0) \mapsto \rf(t)$ is an orientation-preserving function.
If the field lines of $\Bf$ move along with this flow,
then corollary~\ref{cor:hopf-invariant-of-subset-is-invariant}
tells us that the helicity of $\Bf$ in a volume $U$
does not change as $U$ moves along with the flow.
